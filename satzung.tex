%! TEX root = main.tex
\documentclass[11pt,a4paper]{scrartcl}

\usepackage[english]{babel}
\usepackage[backend=biber, bibencoding=utf8, citestyle=numeric-comp, backref=true, hyperref=true, sorting=none]{biblatex}
\usepackage[babel]{csquotes}
\usepackage[nodayofweek]{datetime}
\usepackage[bookmarks, hidelinks]{hyperref}
\usepackage{fancyref}
\usepackage{xcolor}
\usepackage{url}
\usepackage{listings}
\usepackage{caption}
\usepackage{subcaption}
\usepackage{float}
\usepackage{fix-cm} % Allows increasing the font size of specific fonts beyond LaTeX default specifications
\usepackage[innermargin=5.0cm,]{fullwidth}
\usepackage{upquote}% getting the right grave ` (and not ‘)!
\usepackage{xspace}

\newcommand{\RFC}{\texttt{01.RFC Berlin}\xspace}
\renewcommand{\thesection}{\Alph{section}}
\renewcommand{\thesubsection}{\S\arabic{subsection}}

%%%%%%%%%%%%%%%%%%%%%%%%%%%%%%%%%%%%%%%%%%%%%%%%%%%%%%%%%%%%%%%%%%%%%%%%%%%%%
% META DATA
%%%%%%%%%%%%%%%%%%%%%%%%%%%%%%%%%%%%%%%%%%%%%%%%%%%%%%%%%%%%%%%%%%%%%%%%%%%%%

\author{Lutz Freitag \and Michael Pluhatsch}
\title{Satzung des \RFC}

%%%%%%%%%%%%%%%%%%%%%%%%%%%%%%%%%%%%%%%%%%%%%%%%%%%%%%%%%%%%%%%%%%%%%%%%%%%%%
% CONTENT
%%%%%%%%%%%%%%%%%%%%%%%%%%%%%%%%%%%%%%%%%%%%%%%%%%%%%%%%%%%%%%%%%%%%%%%%%%%%%

\begin{document}

\maketitle
\section{Allgemeines}
\subsection{Name, Sitz, Eintragung und Geschäftsjahr}\label{sec:scope}
Der Verein führt den Namen \glqq Roboter Fußball-Club Berlin\grqq.
Der Sitz ist Berlin.
 und 
%ist in das Vereinsregister beim Amtsgericht ................. unter der Nr. ...........eingetragen.
soll in das Vereinsregister eingetragen werden. Nach der Eintragung f\"uhrt er zu seinem Namen den Zusatz e.V. Das Gesch\"aftsjahr ist das Kalenderjahr.

\subsection{Zweck und Aufgaben}~\label{sec:purpose}
\begin{itemize}
    \item[1)] Der  Verein  verfolgt  ausschließlich  und  unmittelbar  gemeinnützige Zwecke  im  Sinne  des Abschnitts \glqq Steuerbeg\"unstigte Zwecke\grqq{} der Abgabenordnung. Zweck und Aufgabe des Vereins ist die F\"orderung von Robotersport in Berlin. Insbesondere verschreibt sich der \RFC dem Roboterfu{\ss}ball. Der Verein ist frei von politischen, rassischen und konfessionellen Bindungen. 
    \item[2)] Der Satzungszweck wird verwirklicht insbesondere durch die Durchführung von allgemeinen Roboter orientierten Veranstaltungen, Schulungen und Weiterbildungen, die Beteiligung an Turnieren, Vorführungen und sportlichen Wettkämpfen. Der Verein kann außer für den Roboterfu{\ss}ball auch andere Roboterabteilungen unterhalten. 
 	\item[3)] Der Club ist selbstlos t\"atig, er verfolgt nicht in erster Linie eigenwirtschaftliche Zwecke. Mittel des Vereins d\"urfen nur f\"r die satzungsmäßigen Zwecke verwendet werden.
 	\item[4)] Die Mitglieder  erhalten  keine  Zuwendungen  aus  Mitteln  des  Vereins.  Es  darf  keine  Person durch Ausgaben, die dem Zweck des Vereins fremd sind, oder durch unverh\"altnismäßig hohe Vergütungen beg\"unstigt werden.
\end{itemize}

% \section{Gemeinn\"utzigkeit}\label{sec:gemeinnutz}

\section{Mitgliedschaft}\label{sec:mitgliedschaft}
\setcounter{subsection}{2}
\subsection{Erwerb der Mitgliedschaft}
\begin{itemize}
    \item[1)] Mitglied des Vereins k\"onnen nat\"urliche und juristische Personen werden.
    \item[2)] Die Mitgliedschaft wird durch Aufnahme erworben. Es ist ein schriftlicher Aufnahmeantrag an den Verein zu richten. 
    \item[3)] Der Aufnahmeantrag eines Minderj\"ahrigen bedarf der schriftlichen Einwilligung der gesetzlichen Vertreter. Mit der Einwilligung wird die Zustimmung zur Wahr\-neh\-mung der Mitgliederrechte  und –pflichten durch das minderj\"ahrige  Mitglied erteilt. Die  gesetzlichen Vertreter der  minderj\"ahrigen Vereinsmitglieder verpflichten sich mit der Unterzeichnung des Aufnahmeantrags f\"ur  die Beitragspflichten des  Minderj\"ahrigen bis zur Vollendung des 18 .Lebensjahrs pers\"onlich gegenüber dem Verein zu haften. 
    \item[4)] \"Uber den schriftlichen Aufnahmeantrag entscheidet das Präsidium. Mit der Abgabe  des unterzeichneten Aufnahmeantrags erkennt das Mitglied die Vereinssatzung und die Ordnungen in der jeweils g\"ultigen Fassung an.
    \item[5)] Ein  Aufnahmeanspruch  besteht  nicht.  Die  Ablehnung  der  Aufnahme  muss  nicht  begr\"undet werden. Ein Rechtsmittel gegen die Ablehnung der Aufnahme besteht nicht.
    \item[6)] Die Mitgliedschaft tritt erst mit Bezahlung der Aufnahmegebühr und mindestens eines Jahresbeitrages in Kraft.
\end{itemize}

\subsection{Arten der Mitgliedschaft}
 \begin{itemize}
    \item[1)] Der Verein hat:
    \begin{itemize}
        \item aktive Mitglieder
        \item passive Mitglieder
        \item Ehrenmitglieder
    \end{itemize}
 \end{itemize}

\section{Teams}\label{sec:teams}

\end{document}
